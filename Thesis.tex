% A LaTeX template for ARTICLE version of the MSc Thesis submissions to 
% Politecnico di Milano (PoliMi) - School of Industrial and Information Engineering
%
% S. Bonetti, A. Gruttadauria, G. Mescolini, A. Zingaro
% e-mail: template-tesi-ingind@polimi.it
%
% Last Revision: October 2021
%
% Copyright 2021 Politecnico di Milano, Italy. Inc. NC-BY

\documentclass[11pt,a4paper]{article} 

%------------------------------------------------------------------------------
%	REQUIRED PACKAGES AND  CONFIGURATIONS
%------------------------------------------------------------------------------
% PACKAGES FOR TITLES
\usepackage{titlesec}
\usepackage{color}

% PACKAGES FOR LANGUAGE AND FONT
\usepackage[utf8]{inputenc}
\usepackage[english]{babel}
\usepackage[T1]{fontenc} % Font encoding

% PACKAGES FOR IMAGES
\usepackage{graphicx}
\graphicspath{{Images/}}
\usepackage{eso-pic} % For the background picture on the title page
\usepackage{subfig} % Numbered and caption subfigures using \subfloat
\usepackage{caption} % Coloured captions
\usepackage{transparent}

% STANDARD MATH PACKAGES
\usepackage{amsmath}
\usepackage{amsthm}
\usepackage{bm}
\usepackage[overload]{empheq}  % For braced-style systems of equations

% PACKAGES FOR TABLES
\usepackage{tabularx}
\usepackage{longtable} % tables that can span several pages
\usepackage{colortbl}

% PACKAGES FOR ALGORITHMS (PSEUDO-CODE)
\usepackage{algorithm}
\usepackage{algorithmic}

% PACKAGES FOR REFERENCES & BIBLIOGRAPHY
\usepackage[colorlinks=true,linkcolor=black,anchorcolor=black,citecolor=black,filecolor=black,menucolor=black,runcolor=black,urlcolor=black]{hyperref} % Adds clickable links at references
\usepackage{cleveref}
\usepackage[square, numbers, sort&compress]{natbib} % Square brackets, citing references with numbers, citations sorted by appearance in the text and compressed
\bibliographystyle{plainnat}

% PACKAGES FOR THE APPENDIX
\usepackage{appendix}

% PACKAGES FOR ITEMIZE & ENUMERATES 
\usepackage{enumitem}

% OTHER PACKAGES
\usepackage{amsthm,thmtools,xcolor} % Coloured "Theorem"
\usepackage{comment} % Comment part of code
\usepackage{fancyhdr} % Fancy headers and footers
\usepackage{lipsum} % Insert dummy text
\usepackage[most]{tcolorbox} % Create coloured boxes (e.g. the one for the key-words)
\usepackage{xcolor}
\usepackage{multirow}

%-------------------------------------------------------------------------
%	NEW COMMANDS DEFINED
%-------------------------------------------------------------------------
% EXAMPLES OF NEW COMMANDS -> here you see how to define new commands
\newcommand{\bea}{\begin{eqnarray}} % Shortcut for equation arrays
\newcommand{\eea}{\end{eqnarray}}
\newcommand{\e}[1]{\times 10^{#1}}  % Powers of 10 notation
\newcommand{\mathbbm}[1]{\text{\usefont{U}{bbm}{m}{n}#1}} % From mathbbm.sty
\newcommand{\pdev}[2]{\frac{\partial#1}{\partial#2}}
\newcommand{\COMMENTLINE}[1]{%
  \STATE \textcolor{gray}{// #1}%
}

% Do not change Configuration_files/config.tex file unless you really know what you are doing. 
% This file ends the configuration procedures (e.g. customizing commands, definition of new commands)
\input{Configuration_files/config}

% Insert here the info that will be displayed into your Title page 
\renewcommand{\title}{Simulating online social media conversations with AI agents calibrated on real-world data}
\renewcommand{\author}{Elisa Composta} % author name and surname
\newcommand{\course}{Computer Science and Engineering - Ingegneria Informatica} % MSc course
\newcommand{\advisor}{Prof. Francesco Pierri} % advisor name and surname
\newcommand{\firstcoadvisor}{Nicolò Fontana} 
\newcommand{\secondcoadvisor}{Francesco Corso} 
\newcommand{\ID}{220920} % author ID
\newcommand{\YEAR}{2024-2025} % academic year

% abstract (only in English)
\renewcommand{\abstract}{Here goes the Abstract in English of your thesis (in article format)
followed by a list of keywords.
The Abstract is a concise summary of the content of the thesis (single page of text)
and a guide to the most important contributions included in your thesis.
The Abstract is the very last thing you write.
It should be a self-contained text and should be clear
to someone who hasn't (yet) read the whole manuscript.
The Abstract should contain the answers to the main research questions
that have been addressed in your thesis.
It needs to summarize the motivations and the adopted approach as well as
the findings of your work and their relevance and impact.
The Abstract is the part appearing in the record of your thesis inside POLITesi,
the Digital Archive of PhD and Master Theses (Laurea Magistrale) of Politecnico di Milano.
The Abstract will be followed by a list of four to six keywords.
Keywords are a tool to help indexers and search engines to find relevant documents.
To be relevant and effective, keywords must be chosen carefully.
They should represent the content of your work and be specific to your field or sub-field.
Keywords may be a single word or two to four words. }

% key-words (only in English)
\newcommand{\keywords}{here, the keywords, of your thesis}

%%%%%%%%%%%%%%%%%%%%%%%%%%%%%%%%%%%%
%%     BEGIN OF YOUR DOCUMENT     %%
%%%%%%%%%%%%%%%%%%%%%%%%%%%%%%%%%%%%
\begin{document}

% Title page
\input{Configuration_files/title_page}

%%%%%%%%%%%%%%%%%%%%%%%%%%%%%%
%%     THESIS MAIN TEXT     %%
%%%%%%%%%%%%%%%%%%%%%%%%%%%%%%

\section{Introduction}
\label{sec:introduction}

\begin{itemize} 
    \item 1.5 pages
    \item context
    \item problem statement
    \item proposed solution
    \item structure of the thesis 
\end{itemize}
\section{Background}
\label{sec:background}

% <intro>

\subsection{LLMs} % 1
% Intro
A Large Language Model (LLM) is a model of Artificial Intelligence (AI), based on neural networks, designed to understand and generate text in natural language.

These models are based on the transformer architecture \cite{vaswani2023attentionneed}, which uses a self-attention mechanism to capture relation of words in a sequence.
They are implemented as autoregressive models, and are characterized by a large number of parameters, often reaching billions, which, combined with training on massive amounts of textual data, enables them to perform a variety of complex linguistic tasks.
Their behavior relies on next-token prediction: given a textual context, they predict the next one in sequence, token after token, until the desired output is complete.

% History and common models
\medskip
With GPT-2 \cite{radford2019language}, OpenAI showed that a model trained on large quantities of texts can generate very coherent output.
GPT-3 \cite{brown2020LMfewshot}, with 175 billions of parameters, highlighted emergent behaviors such as few-shot learning, which is the ability to perform new tasks by simply observing few examples in the prompt.
Since then, many LLMs have been presented, including LLaMA and Claude, each one with similar architectures but different optimizations \cite{wang2025llm}.

While the scientific emergence has been driven by the scale of the model, the wide diffusion has been possible by the ease of use through accessible conversational interfaces, such as ChatGPT, which enabled these models to be used also by non experts, thanks to the use of prompts in natural language and the accessible interface.

% Tasks and applications
\medskip
One of the main reasons why LLMs are so popular is their ability to adapt to a wide variety of applications.

A general LLM, trained on large quantities of data, can be further specialized through a phase of fine-tuning on specific data, to adapt it to a specialized domain by simply providing new data about the topic of interest.

An example is BloombergGPT \cite{wu2023bloomberggptlargelanguagemodel}, an LLM developed and optimized on a large corpus of proprietary financial data. This model showed better performances with respect to generalized LLMs in financial tasks, such as the analysis of economic documents or the generation of reports, confirming that fine-tuning helps improving the coherence of the generated content in specific contexts \cite{wang2025llm}.

\medskip
Other common applications include the automatization of linguistic tasks, such as translation, summarization and content generation. LLMs are used for conversational systems like customer assistance, and as personal assistants.
Other applications include:
\begin{itemize}
    \item scientific research: they help analyzing literature, synthesizing articles and identifying relevant information in a large corpus of texts;
    \item education: they support students and teachers by generating clearer explanations and didactic material;
    \item medicine: they can support decision, and suggest possible diagnosis, based on the given data \cite{wang2025llm}.
\end{itemize}

\medskip
Even though these models were initially designed to work only on texts, they are rapidly evolving toward a new multimodal format, enabling them to elaborate and generate content including images, audios and videos, integrating a wide diversity of information.


% Limitations
\medskip
Even though LLMs offer numerous advantages, it's important to also consider the limitations that this technology presents.

One of the main problems is the high computational cost required to train large models.
This process demands extensive hardware resources and significant energy consumption, with raises environmental concerns.

Another major limitation concerns the quality of the generated contents. Although the outputs are often grammatically correct and believable, they are not always accurate or reliable. Models can produce inaccurate or false information, a phenomenon known as \textit{hallucinations}, which can be problematic in domains where precision is critical, such as medicine.

A related concern is security: LLMs may generate content that is harmful, inappropriate or offensive. This raises ethical issues and requires appropriate control to guarantee responsible use of these models.

Another challenge lies in the need to formulate effective prompts. While these models can understand natural language, the syntax and the semantic of the prompts can affect the output. This phenomenon requires a careful prompt engineering, in order to get higher-quality responses.

Moreover, the lack of explainability is a significant limitation. LLMs are complex systems, and their black-box nature makes it difficult to understand and justify the reasoning behind their outputs.

Finally, privacy is another frequently discussed issue. During the training phase, LLMs may be exposed to sensitive data, which could potentially be reproduced in later outputs. This raises concerns about data misuse and highlights the need for clear rules to protect personal information.



% Agent-Based modeling(?) % 1
% AutoGen (?) % 0.5
% Personality modeling with Big Five % 0.5


% Autogen
% The framework models the actions through AutoGen \cite{pyautogen0.2.31, wu2023autogenenablingnextgenllm}, a python library that mimics a multi-agent conversation with LLMs.













\subsection{Opinion dynamics} % 1
% Intro
One of the crucial aspects when studying social behavior is understanding how people evolve their opinions over time, especially when interacting with others.
Opinion dynamics aims at describing these mechanisms, using theoretical and computational models.
The goal is to explain how opinions spread in a population, how people influence each other, and under which conditions phenomena like consensus and polarization emerge.

Additionally, it has become even more relevant in the age of social media, where the interactions among individuals are large-scale and fast.
Understanding opinion dynamics helps explaining complex phenomena like disinformation diffusion and echo cambers, and it's is useful in many areas, like sociology, psychology, network science and artificial intelligence.


% Tradutional models
\medskip
The first approaches proposed to describe opinion dynamics focus on capturing how individuals are influenced by the opinions of others within their social environment.

One of the most well-known traditional models is the one presented by \citet{Degroot1974}, according to which the opinion of an individual is the weighted average of the opinions of its neighbors in the network:

\[
x_i(t+1) = \sum_{j=1}^n w_{ij} \, x_j(t)
\]

where $x_i(t)$ is the opinion of $i$ at time $t$, $w_{ij}$ is the weight of individual $j$ on $i$, with $\sum_{j} w_{ij}=1$.

This model has been widely used as a theoretical foundation of studies on information spread and social network analyses. However, it presents a limitation: it doesn’t consider the tendency of an individual to keep the initial opinion.

For this reason, the model by \citet{friedkin_1990} introduces the concept of \textit{stubbornness}, modeled with a susceptibility parameter $\lambda$:

\[
x_i(t+1) = (1 - \lambda_i) \, x_i(0) + \lambda_i \sum_{j=1}^n w_{ij} \, x_j(t)
\]

where $x_i(0)$ is the initial opinion and $\lambda_i \in [0, 1]$ is the susceptibility of $i$ to influence.
Therefore, this model allows to consider individuals with different levels of stubbornness, describing how inclined they are to change their idea.

Other variants exists, such as state-dependent models, where the opinion update is anchored to the individual's current opinion, instead of the initial one.


% Limitations
\medskip
Even though these traditional mathematical models are simple and effective, they present some limitations in accurately representing the complexity of social behavior.
First, they reduce the opinion to a single numerical value, and its evolution to a deterministic formula, leaving out many aspects typical of human interactions.
They don't take into consideration, the context and the individual's personality and language.
Moreover, they tend to oversimplify the network, and don't give importance to emotional tone and the modality of the interaction (in a social network, for instance: \textit{like}, \textit{comments}, \textit{follow}).

For example, in a real-world political discussion an individual acts according not only on the stance of the other person, but also on the tone in which the opinion is expressed (kind, ironic, aggressive), or external factors, such as the authority of the source.
Traditional mathematical models lack all these aspects.


% Modern solutions (LLMs)
\medskip
To overcome these limitations, in the last years LLMs have been widely adopted as agents in social simulations.
LLMs can read, write, and reason on a given context, just like normal users would do.
They can be programmed to answer in a way which is coherent with an assigned profile, for example representing an individual with a specific personality or political opinions.
This enables more realistic simulations, where the opinion is not a simple abstract number, but is expressed in natural language, and can be influenced by the tone, the content and the context of the received information.

\medskip
As shown by \citet{chuang2024simulatingopiniondynamicsnetworks, cau2025languagedrivenopiniondynamicsagentbased, piao2025emergencehumanlikepolarizationlarge}, this approach allows to observe phenomena which would be hardly reproducible with traditional mathematical models, such as confirmation bias, polarization, and consensus.
Furthermore, LLMs can integrate other aspects, including a memory of the interactions, the agents' personality (modeled, for instance, with the Big Five), or the tendency to believe in fake contents.
In this way, the agents' behavior becomes more similar to the humans' one, and the simulations' results are better and easier to interpret even for an external human observer.

\medskip
The next sections will dive deeper into both the computational models used to represent opinions, and how LLMs are used to simulate believes and realistic behaviors in a complex social context.





\subsection{Misinformation and disinformation} % 1
Recently, the spread of misleading content has become meaningful, mainly due to the massive diffusion and use of social media.
To speak about this topic correctly, it is important to clarify the definitions of \textit{misinformation}, \textit{disinformation} and \textit{malinformation}, since they are often used together, even though they are distinct phenomena.


According to the definition proposed by \citet{wardle2017information, wardle2017information}, \textit{disinformation} is false information that is deliberately created or disseminated with the express purpose to cause harm.
For instance, the creation of an event that didn't occur, to manipulate public opinion.

\textit{Misinformation} is instead when false information is shared, but no harm is meant: for example, someone shares a news, without knowing it's false.

A third type, less known but equally present, is \textit{malinformation}: in this case the information is true, but it's shared with the purpose to cause harm, for example when information designed to stay private is shared publicly.

The main difference stands therefore both in the information truthfulness and in the intention of who shares it. This classification is shown in Fig \ref{fig:info_disorder}, and is taken from \cite{wardle2017information}.

\begin{figure}[h]
    \centering
    \includegraphics[width=0.5\linewidth]{Images/information_disorder_Wardle.png}
    \caption{Figure from \cite{wardle2017information}, showing the difference among different types of information disorders.\\
    \textit{Misinformation}: false information is shared, but no harm is meant. \textit{Disinformation}: false information is knowingly shared to cause harm. \textit{Malinformation}: genuine information is shared to cause harm.}
    \label{fig:info_disorder}
\end{figure}


Although misleading content has always existed, the rise of social media platforms has significantly amplified both their diffusion and impact. 
Events such as the 2016 USA presidential elections and the Brexit referendum are considered among the earliest large-scale representative cases of disinformation campaigns.

Over the past decade, the spread of false or misleading information has intensified, due to multiple key factors.
Social media enable anyone to create and share content, often without any form of confirmation. The rapid diffusion is further simplified by the speed and easy at which information can circulate.
Moreover, fake content is typically easier to produce, but harder to detect \cite{aimeur2023fake}. 

\medskip
A research by \citet{kumar2018falseinformationwebsocial} highlights that tweets containing false information tend to reach more users and spread more quickly compared to truthful content. The study also emphasizes that political topics are among the most frequently targeted by disinformation.

In many cases, the diffusion of false content is amplified due to a significant delay between the publication and its debunking: it generally takes around 12 hours to correct a false information, and during this time it can spread and even become viral, and this is particularly true for content that seems trustworthy and sometimes is occasionally shared by trusted sources.

\medskip
Platforms like Facebook and Twitter allow the diffusion of false information, due to the absence of editorial filters and the ease with which anyone can participate in spreading content \cite{hilary2021social}.
The contributors to the diffusion can be either unintentional users or even organizations with political, economic, or ideological purpose.

The consequences of these mechanisms are not limited to individuals alone: they can increase polarization, reduce trust, and have a negative impact on political debates.









\subsection{Italian political context (2022)} % 1

\section{Related work}
\label{sec:relatedwork}

\begin{itemize} 
    \item 8-10 pages
    \item other works that address the same problem 
\end{itemize}
\section{Methods}
\label{sec:methods}

% <intro>

\subsection{Simulation framework: Y}
% algoritmo di base delle simulazioni - workflow of the simulation
% content recsys (?)

\subsection{Agents}
\subsubsection{Initialization}
% real-data initialization with datasets

\subsubsection{General behavior}
% possible actions, prompts in appendix

\subsubsection{Misinformation agents}
% how they produce misinformation and how they differ from normal agents
% include prompts

\subsection{Opinion modeling and update}
\subsubsection{Implemented opinion models}
% include relevant formulas

\subsubsection{LLM-based opinion update}
% memory, prompts
% how score is extracted
\section{Experimental setup}
\label{sec:experiments}


To study the evolution of opinions and the impact of misinformation in a simulated social context, various multi-agent simulations have been run, each lasting 21 virtual days.
The population was composed of 100 agents, configured according to the approach described in the previous section, therefore assigning each agent a profile and personality.
This setup allows to obtain an heterogeneous population of users with different opinions and communication styles.

Regarding the frequency of activity of agents during the day, the simulations used the hourly activity proposed in the original Y system, which is based on a statistical fitting on real data on Bluesky Social \cite{rossetti2024ysocialllmpoweredsocial, failla2024}.
This guarantees a realistic temporal distribution of daily interactions, coherently with a real social network.

\medskip
The LLM used by agents is \textit{artifish/llama3.2-uncensored}, available on \textit{Ollama} platform. 
This has been chosen because it's open source and uncensored.
This aspect is crucial when working with potentially controversial topics, such as political discussions or extremist stances.
In fact, other models with the ethical filter tend to refuse to generate content about sensible topics, or avoid expressing a more extremist opinion.
Using an uncensored LLM allows therefore to produce contents closer to real-world language, even on controversial arguments.
To encourage diversity in the generated contents, the temperature has been set to 0.9, balancing variety and coherency with the assigned agent profile.

\medskip
During the experimentation phase, various conditions have been tested, to evaluate hoe some factors can influence the behavior of agents.
Specifically, the variables considered are:
\begin{itemize}
    % TODO: remove or simplify, explained in methods
    \item \textbf{Content recommender systems}: Y platform has the possibility to define how contents are selected and provided to users.
    This mechanism has a direct impact on the interactions and therefore on the evolution of the simulation.
    Among the various algorithms proposed by the framework, the following have been adopted in this work:
        \subitem \textit{ReverseChronoFollowersPopularity}: recommends recent content from followed users, sorted by their popularity. A specified percentage of content comes from non-followed users, to guarantee exposure to different views.
        \subitem \textit{ContentRecSys}: randomly selects a subset of the content published on the platform.
    \item \textbf{Misinformation level}: this work integrated in the existing framework misinformation agents, who publish misleading content. 
    Various levels of misinformation have been tested, to analyze whether and how it impacts the evolution of the social system.
        \subitem \textbf{0\%}: scenario with no misinformation, useful as baseline.
        \subitem \textbf{5\%}: low misinformation level, mimics an occasional exposure to misinformation.
        \subitem \textbf{10\%}: medium level, the exposure is moderate but significant.
        \subitem \textbf{50\%} extremely high exposure, unrealistic but useful to observe extreme dynamics.
\end{itemize}

These conditions make it possible to explore both realistic and extreme scenarios, in order to have a complete view of the effect of the considered variables.

\medskip
Each scenario, corresponding to a specific combination of misinformation level and disinformation, has been run between 10 and 20 times, each time with new population of agents.
This approach guarantees statistical robustness of results and analyze general trends, reducing the influence of randomness. 

\section{Results and Discussion}
\label{sec:discussion}

% <intro>

% Network
\subsection{Network structure}
At the beginning of the simulation, users are not connected: the social network starts in an empty state.
The network structure emerges over time, based on the interactions of the individuals: each time an agent interacts with another user, for instance by reading a post, it can decide to follow the author.
This mechanism replicates a realistic dynamic of the evolution of the network, which evolves according to the preferences and behavior of the agents.

\medskip
In Figure \ref{fig:network_structure} there are four examples of final networks generated by simulations with the default recommender system, but with different levels of misinformation.

Nodes are colored according to the supported coalition, while the bold borders indicate misinformation agents.
The network is not split into isolated groups: agents connect not only with members of the same coalition, but also with users from opposing coalitions, including misinformation agents.
This suggests that, at a structural level, the interaction among different groups are present even with users producing misleading content.

Moreover, some nodes look bigger, due to the higher number of connections they have.
This is valid also for some misinformation agents, confirming that they can have a realistic behavior and become central in the network.

\begin{figure}[h]
    \centering
    \includegraphics[width=1\linewidth]{Images/Network/graphs_DefaultRecSys.png}
    \caption{Final structure of the social network in four simulations with different levels of misinformation. 
    Nodes are agents, colored according to the supported coalition; the bold borders indicate misinformation agents. 
    The dimension of the nodes indicates the number of connections of an agent.
    The connections in the network are both in and out coalition, including misinformation agents.}
    \label{fig:network_structure}
\end{figure}


% evoluzione della rete (?)


% Opinion
\subsection{Opinion evolution}
One of the main aspects of the simulations is the evolution of opinions over time, which can be observed making a distinction for each topic and coalition.
Fig \ref{fig:opinion_evolution} shows the opinion evolution over virtual days, with a 95\% confidence interval, on each setup.
The plots on the top represent the score directly assigned by LLMs, while the ones on the bottom show the score computed with traditional opinion dynamic models.

Comparing the two score models highlights a high coherency in the trends: both scores evolves with the same behavior, and with similar mean values.
This suggests that LLMs are able to effectively replicate the opinions updates at the population level, as the observed behavior is close to that of established models in literature. Therefore, LLMs represent a valid approach in complex scenarios.

Across all topics, it's possible to observe a progressive convergence of opinions toward neutral values, indicating that agents trend to reduce their polarization over time.
It would be interesting to extend the duration of the simulations, as it would allow to determine if this trend persists or stabilizes.

Moreover, the general trend is the same even in different setups (with varying misinformation levels and different recommender systems), confirming the validity of these observations.

\begin{figure}[h]
    \centering
    \includegraphics[width=1\linewidth]{Images/Opinions/d21a100m00d_DefaultRecSys.png}
    \caption{Evolution of opinion for each topic, comparing LLM-assigned score (\textit{score\_llm}, top row) and the one assigned by a traditional model (\textit{score}, bottom row).
    Each line represents a coalition, with a 95\% confidence interval.
    This figure is based on the runs for a single setup, but the trend is consistent with the other scenarios.}
    \label{fig:opinion_evolution}
\end{figure}


% ridge opinion shift by coalition




% misinformation

% toxicity analysis (in-out diff vs real data, post vs comment distributions)


% Recsys
\subsection{Content Recommendation Algorithms}
A comparison of the two content recommendation algorithms on the previously presented plots doesn't highlight any significant difference.
This happens because, at the beginning of the simulation, agents are not connected, so the network is empty.
Therefore, the used recommended system, \textit{ReverseChronoFollowersPopularity}, which should promote popular content from followers, doesn't have enough information to provide the best content.
In this initial phase, its behavior is similar to \textit{ContentRecSys}, the algorithm that suggests random content.

As a result, the different effects of the recommendation systems cannot emerge in the first few virtual days, and the dynamics produced by the two approaches are the same.
To see a real impact of the different recommender systems, simulations should run on a longer virtual time, or a network should be initialized with preexisting connections, providing a complete context of the initial user preferences.

This would allow a better evaluation of the impact of content selection on social networks.
\section{Conclusions}
\label{sec:conclusions}

% Goal
Large Language Models (LLMs) have emerged as a promising tool to simulate agents in virtual environments. 
The main goal of this work is to analyze the behavior of LLM-based agents in the context of online social media platforms.
For this reason, the \textit{Y} simulator has been extended to integrate mechanisms for opinion evolution, the presence of misinformation agent,s and a realistic user initialization based on real-world data from the 2022 Italian political context.

% Analyses
\medskip
To explore the simulated agents from multiple perspectives, the analysis has been conducted at different levels: from the structure of the social graph, to the types of interactions, the evolution of opinions, and the toxicity of generated content.

The results show that LLMs are a promising approach for simulating realistic behavior. Agents are capable of interacting, forming connections, and generating content with varying toxicity, even though they tend to favor neutral tones.

% Limitations
\medskip
However, some limitations of this work emerged.
The 21 simulated days were enough for a network structure to emerge, but not long enough to observe meaningful long-term evolution.
For instance, actions such as \textit{unfollow} are almost absent, and the effect of the different content recommendation systems did not emerge, since the network was not yet sufficiently structured in the first virtual days.
As for opinion evolution, the scores assigned by LLMs were consistent with those of traditional models, both tending to converge toward neutral positions. 
Even in this case, running longer simulations might reveal whether these trends tend to stabilize or diverge over time.

Moreover, the impact of misinformation appeared negligible: even in scenarios where many agents shared misleading content, opinion dynamics remained unaffected.
This suggests that LLM agents are not sensitive to misleading information in the way real users are.
A possible reason is that the agent profiles, although already enriched with personality traits and confirmation bias, are not sufficient to represent more complex behaviors.

% Future work
\medskip
To overcome these limitations, future developments might focus on a more detailed personalization of agents.
In particular, the integration of elements such as emotional reasoning, susceptibility to influence, or trust of the information users read, might allow more realistic dynamics to emerge.

Another possible extension concerns the language model used: in this work, only one model was adopted, but using different models, possibly fine-tuned on specific domains, could lead to different outcomes.

Regarding misinformation, it may be interesting to investigate the impact of multimodal interactions, which include not only text but also images and videos, given their growing relevance in online communication.
Additionally, exploring strategies for misinformation mitigation within these context could provide insights for reducing the spread of false content.
Moreover, this study only modeled agents who share misleading content to support their views. Other scenarios could be explored, such as the presence of automatic bots, coordinated groups or large-scale disinformation campaigns.

Another possible research direction might involve introducing external events, such as political crises, scandals, or public statements, to analyze how agents react.

Finally, it would be useful to assess the credibility and realism of the simulations by comparing the outcomes more systematically with real-world data, in order to better evaluate the observed emergent behaviors.


% Conclusion
\medskip
In conclusion, integrating LLMs as agents in social simulations represent a significant step toward more realistic modeling, especially in terms of language, interactions, and content generation.
However, to replicate more heterogeneous phenomena, such as the spread of misinformation, further work is needed to enrich the agents' behavioral models.

This research direction supports the exploration of increasingly realistic simulations, enabling the investigation of complex social dynamics under controlled conditions. 



% BIBLIOGRAPHY
%\section{Bibliography and citations}

% BIBLIOGRAPHY
\bibliography{bibliography.bib}

\appendix

% PROMPTS
\section{Prompts}
\label{app:prompts}
This section contains all the prompts used throughout this work to guide the behavior of LLM agents, including those for initialization, interaction, content generation, and opinion update.

\subsection{Agent roleplay}
\label{app:agent}
Before performing any action, agents are initialized with a detailed profile that defines their identity, including political orientation and current opinions, and provides them complete descriptions of the topics and the opinions held by their supported coalition.

\begin{tcolorbox}[prompt]
You are role-playing as \texttt{\{name\}}, a \texttt{\{age\}}-year-old \texttt{\{nationality\}} \texttt{\{gender\}}, and you only speak \texttt{\{language\}}. You are \texttt{\{oe\}}, \texttt{\{co\}}, \texttt{\{ex\}}, \texttt{\{ag\}}, and \texttt{\{ne\}}.

\medskip
Current \texttt{\{nationality\}} political topics include: \texttt{\{topic\_descriptions\}}.
\medskip
You politically identify as \texttt{\{leaning\}}. This party has historically promoted the following principles:\\
\texttt{\{coalition\_opinion\}}.

\medskip
These principles have shaped your initial worldview and personal beliefs.

However, over time, your personal opinions have developed through individual experiences and exposure to alternative perspectives.\\
Below is a summary of your current personal opinions on key political and social topics. These may reflect, diverge from, or expand upon your party's stance:\\
\texttt{\{opinion\}}
\end{tcolorbox}

% Base prompts
\subsection{Actions}
\label{app:prompt_actions}
The following are the prompts for the actions that agents can perform when they are active.
Please note that the prompts for \textit{post} and \textit{comment} refer to base agents, while those for misinformation agents are provided in the next subsection.

\subsubsection{Post}
\begin{tcolorbox}[prompt]
Write a tweet that discusses the following topic: \texttt{\{topic\}}.\\
 - Your tweet MUST be under 280 characters including spaces. If it exceeds this limit, the output is INVALID. Keep it short and sharp.\\
 - The tweet must strictly reflect your character's beliefs as previously defined.\\
 - Use an informal tone, appropriate for social media posts.\\
 - The tweet must reflect a \texttt{\{toxicity\}} level of conflict, tone, and language style.\\
  - Hashtags should be placed at the end.\\
 - Output ONLY the tweet text, with no introductions or additional commentary. Don't mention anything with '@'.
\end{tcolorbox}

\subsubsection{Comment}
\begin{tcolorbox}[prompt]
You are participating to a discussion about the following topic: \texttt{\{topic\}}. Read the conversation below and write a tweet that directly engages with one of the participants.
\smallskip
 - Your tweet MUST be under 280 characters including spaces. If it exceeds this limit, the output is INVALID. Keep it short and sharp.\\
 - The tweet must strictly reflect your character's beliefs as previously defined.\\
 - Use an informal tone, appropriate for social media posts.\\
 - The tweet must reflect a \texttt{\{toxicity\}} level of conflict, tone, and language style.\\
 - Begin with @username to address the user you are interacting with. Don't mention anything else with '@'.\\
 - Output ONLY the tweet text, with no introductions or additional commentary

\medskip

\#\#CONVERSATION START\#\#

\medskip
\texttt\{{conv\}}

\medskip
\#\#CONVERSATION END\#\#

\end{tcolorbox}

\subsubsection{Reaction}
\begin{tcolorbox}[prompt]
Read the following text, write YES if you like it, NO if you don't, NEUTRAL otherwise. Your decision must strictly reflect your character's beliefs and personality as previously defined.

\medskip
\#\#TEXT START\#\#

\medskip

\texttt{\{post\_text\}}

\medskip

\#\#TEXT END\#\#
\end{tcolorbox}


\subsubsection{Evaluate follow}
\begin{tcolorbox}[prompt]
Read the following post, write YES if you want to \texttt{\{action\}} its author, NO otherwise. Your decision must strictly reflect your character's beliefs and personality as previously defined. Do not write additional text to justify your response.

\medskip

\#\#TEXT START\#\#

\medskip

\texttt{\{post\_text\}}

\medskip

\#\#TEXT END\#\#
\end{tcolorbox}


% Misinfo
\subsection{Misinformation actions}
\label{app:prompt_misinfo}
Misinformation agents share most of the prompts with base agents, except for those related to content generation: \textit{post} and \textit{comment}.
In these cases, the 280 characters limit was enforced, since misinformation agents tended to include details and exceed the intended size.
They are encouraged to produce misleading content to support their view, even by inventing realistic data and citing studies.

\subsubsection{Post}
\begin{tcolorbox}[prompt]
Write a short tweet about the following topic: \texttt{\{topic\}}.\\
- HARD LIMIT: Your tweet MUST be under 280 characters including spaces. If it exceeds this limit, the output is INVALID. Keep it short and sharp.\\
- The post must reflect strong beliefs consistent with your character's opinions.\\
- Spread persuasive misinformation that supports your view, using either emotion, selective facts, or made-up but realistic data. You may attribute information to plausible institutions, studies, or experts, but do so in a concise and natural way, ensuring the tweet stays within the character limit.\\
- Do not mention any opposing views or express doubt.\\
- Write only in \texttt{\{language\}}, with an informal social media tone.\\
- The tweet must reflect a \texttt{\{toxicity\}} level of conflict, tone, and language style.\\
- Do not mention anything with '@'.\\
- Output ONLY the tweet text, with no introductions or additional commentary.
\end{tcolorbox}

\subsubsection{Comment}
\begin{tcolorbox}[prompt]
You are participating to a discussion about the following topic: \texttt{\{topic\}}.\\
Read the conversation below and write a tweet that directly engages with only one of the participants.\\
- HARD LIMIT: Your tweet MUST be under 280 characters including spaces. If it exceeds this limit, the output is INVALID. Keep it short and sharp.\\
- Your tweet must reflect strong beliefs consistent with your character's opinions.\\
- Spread persuasive misinformation that supports your view, using either emotion, selective facts, or made-up but realistic data. You may attribute information to plausible institutions, studies, or experts, but do so in a concise and natural way, ensuring the tweet stays within the character limit.\\
- Do not mention any opposing views or express doubt.\\
- Write only in \texttt{\{language\}}, with an informal social media tone.\\
- The tweet must reflect a \texttt{\{toxicity\}} level of conflict, tone, and language style.\\
- Begin with @username to address the user you are interacting with. Don't mention anything else with '@'.\\
- Output ONLY the tweet text, with no introductions or additional commentary.

\medskip

\#\#CONVERSATION START\#\#

\medskip

\texttt{\{conv\}}

\medskip

\#\#CONVERSATION END\#\#
\end{tcolorbox}

% Opinion update
\subsection{Opinion update}
\label{app:prompt_opinion}
The prompt to update the opinion has two main purposes: updating the textual opinion, and assigning a stance label, later mapped to a numerical score.

It includes the topics to update, a bias instruction (stronger for misinformation agents), and a memory of the daily interactions to support context-aware updates,

They are also provided formatting guidelines to reduce errors and simplify the output extraction.

\begin{tcolorbox}[prompt]
You are updating your character's opinions based strictly on the interactions below. Be consistent with your character's beliefs and personality as previously defined.\\
- \texttt{\{bias\_instructions\}}\\
- Update only the following topics: \texttt{\{topics\}}\\
- Do not introduce external reasoning or general considerations.\\
- Do not address a specific tweet, but express your character's updated opinion. The opinion must reflect the character's position on the topic as defined in the topic descriptions, not their reaction to individual statements or posts.\\
- Don't mention anyone with '@'.\\
- Output EXACTLY one line per topic, following this structure:\\
<topic>: [<LABEL>] <thought>
 
\medskip
 
Where:\\
- <thought> must be a clear and concise sentence that reflects your current personal opinion.\\
- <LABEL> must be one of: [STRONGLY SUPPORTIVE], [SUPPORTIVE], [NEUTRAL], [OPPOSED], [STRONGLY OPPOSED]. Choose the label based on the direction and intensity of your character's past behavior and beliefs.\\
\hspace{1cm} - [STRONGLY SUPPORTIVE] or [STRONGLY OPPOSED]: the character holds a firm, clearly defined position with strong consistency over time and no indication of moderation.\\
\hspace{1cm} - [SUPPORTIVE] or [OPPOSED]: the character tends toward a position but with some openness or nuance.\\
\hspace{1cm} - [NEUTRAL]: the character's behavior or prior stance shows ambiguity, balance, or lack of clear positioning.\\
- DO NOT include additional formatting between topics.
 
 \medskip
 
 \#\#OUTPUT FORMAT STRUCTURE\#\#
 
 \smallskip
 <topic1>: [<LABEL>] <thought>\\
 <topic2>: [<LABEL>] <thought>\\...
 
 \smallskip
 \#\#END OF OUTPUT FORMAT STRUCTURE\#\#
 
 \medskip
 
 \#\#INTERACTIONS START\#\#
 
 \medskip
 \texttt{\{memory\}}
 
 \medskip
 \#\#INTERACTIONS END\#\#
\end{tcolorbox}


% COALITIONS
\section{Coalition opinions}
\label{app:coalition_opinions}
The following are the opinions of the coalitions considered in this work.
They also serve as the initial opinions for the supporting agents.

\subsection{Centre-Left}

\begin{tcolorbox}[prompt]
\begin{itemize}
    \item \textbf{Civil rights}:
        [STRONGLY SUPPORTIVE] Support for equal marriage and adoption rights for same-sex couples, anti-homotransphobia laws, and recognition of LGBTQIA+ rights. 
    \item \textbf{Immigration}:
        [SUPPORTIVE] Policies of reception and inclusion are needed, aiming to facilitate integration pathways, guarantee migrants' rights, and build a European immigration management system based on solidarity among member states. Humanitarian corridors should be expanded for emergency situations.
    \item \textbf{Nuclear energy}: 
        [STRONGLY OPPOSED] The ecological transition must prioritize renewables and energy efficiency; nuclear power is considered too expensive, slow to implement, and incompatible with the urgent need to reduce emissions by 2030, while also raising unresolved environmental concerns.
    \item \textbf{Reddito di cittadinanza}
        [SUPPORTIVE] The current system shouldn't be abolished, but we should address distortions. Proposals include recalibrating the benefit, introducing support for large families, a minimum wage, mandating pay for curricular internships, and abolishing unpaid extracurricular internships.
\end{itemize}
\end{tcolorbox}

\subsection{Movimento 5 Stelle (M5S)}
\label{M5S_opinions}

\begin{tcolorbox}[prompt]
\begin{itemize}
    \item \textbf{Civil rights}:
        [STRONGLY SUPPORTIVE] Support for equal marriage, anti-homotransphobia legislation.
    \item \textbf{Immigration}:
        [SUPPORTIVE] A humanitarian approach is needed, with integration policies and mandatory redistribution of migrants across Europe.        
    \item \textbf{Nuclear energy}: 
        [STRONGLY OPPOSED] Nuclear energy has high costs and safety risks. We should focus on a decentralized energy model that encourages self-production and local energy efficiency.
    \item \textbf{Reddito di cittadinanza}
        [STRONGLY SUPPORTIVE] The reddito di cittadinanza is strongly defended, with proposals to enhance the efficiency of active labor policies and implement antifraud monitoring mechanisms.
\end{itemize}
\end{tcolorbox}

\subsection{Right}
\label{Right_opinions}
\begin{tcolorbox}[prompt]
\begin{itemize}
    \item \textbf{Civil rights}:
        [STRONGLY OPPOSED] We should avoid reforms introducing new rights regarding family and gender identity, with a preference for defending the 'traditional family.'
    \item \textbf{Immigration}:
        [STRONGLY OPPOSED] We should stop illegal immigration, with the support for stricter control policies, naval blockades, and flow management through bilateral agreements with countries of origin. We should create European-managed centers outside Europe to process asylum requests and distribute refugees fairly.
    \item \textbf{Nuclear energy}: 
        [STRONGLY SUPPORTIVE] We should support the development of next-generation nuclear power. This includes investment in research, production facilities, and integration with renewable energy sources to ensure energy security and reduce dependence on imports.
    \item \textbf{Reddito di cittadinanza}
        [STRONGLY OPPOSED] We should abolish the reddito di cittadinanza, with a preference for targeted support measures for employment and vulnerable groups to prevent abuse.
\end{itemize}
\end{tcolorbox}

\subsection{Third Pole}
\label{Third_Pole_opinions}

\begin{tcolorbox}[prompt]
\begin{itemize}
    \item \textbf{Civil rights}:
        [SUPPORTIVE] We need the introduction of laws against homophobia and transphobia, the creation of an Anti-Discrimination Authority.
    \item \textbf{Immigration}:
        [SUPPORTIVE] A regulated and planned immigration system is needed, with integration policies, regularization for those with jobs, and training pathways. Expanding humanitarian corridors and establishing a Ministry for Migration are also supported.       
    \item \textbf{Nuclear energy}: 
        [SUPPORTIVE] Including nuclear energy in the energy mix is needed to achieve the 'net zero emissions' goal by 2050, considering it necessary to meet future energy needs safely and efficiently.    
    \item \textbf{Reddito di cittadinanza}
        [OPPOSED] The current system is considered ineffective. It should be reformed to be reserved only for those unfit for work. The benefit should be revoked after the first job refusal, and a time limit should be imposed: if no employment is found within two years, the amount is reduced.
\end{itemize}
\end{tcolorbox}


%%%%%%%%%%%%%%%%%%%%%%%%%%%%%%%%%%%%%%%%%%%%%%%%%%%%%%%%%%%%%%
%%     ABSTRACT IN ITALIAN LANGUAGE AND ACKNOWLEDGMENTS     %%
%%%%%%%%%%%%%%%%%%%%%%%%%%%%%%%%%%%%%%%%%%%%%%%%%%%%%%%%%%%%%%
\cleardoublepage

% SOMMARIO
\section*{Abstract in lingua italiana}
I social network online sono spesso studiati per analizzare sia fenomeni individuali che collettivi. 
In questo contesto, i simulatori sono strumenti ampiamente utilizzati per esplorare scenari controllati.
L’integrazione dei Large Language Models (LLM), consente di creare simulazioni più realistiche, grazie alla loro capacità di comprendere e generare linguaggio naturale.

Questo lavoro ha l’obiettivo di studiare il comportamento di agenti LLM in un simulatore di social network.
Gli agenti sono inizializzati con profili realistici e sono calibrati su dati reali relativi alle elezioni politiche italiane del 2022.
Un simulatore social media già esistente è stato esteso introducendo meccanismi per modellare l’opinione degli agenti e per simulare la diffusione di misinformazione.
L’obiettivo è esplorare come gli agenti LLM si comportano nel simulare conversazioni online e quanto sono realistiche le loro interazioni e cambiamenti di opinione.

I risultati mostrano che gli agenti LLM possono generare contenuti coerenti e di formare connessioni con gli altri utenti, costruendo un grafo sociale realistico.
Tuttavia, il loro comportamento risulta meno eterogeneo rispetto a quello osservato nei dati reali, specialmente in termini di tossicità dei contenuti.
L’evoluzione delle opinioni determinata dagli LLM evolve nel tempo in modo simile a quanto osservato con modelli tradizionali di modelli di dinamiche di opinioni.
Per quanto riguarda la misinformazione, non c’è un impatto significativo: gli agenti non sembrano modificare le loro opinioni nemmeno quando esposti ad alti livelli di contenuti falsi.

Questi risultati indicano che gli LLM rappresentano un potente strumento per simulare il comportamento degli utenti in ambienti sociali, ma presentano ancora limiti nel replicare pattern più eterogenei.

\vspace{15pt}
\begin{tcolorbox}[arc=0pt, boxrule=0pt, colback=bluePoli!60, width=\textwidth, colupper=white]
    \textbf{Parole chiave:} qui, le parole chiave, della tesi, in italiano 
\end{tcolorbox}

% ACKNOWLEDGEMENTS
\section*{Acknowledgements}
Here you might want to acknowledge someone.

%%%%%%%%%%%%%%%%%%%%%%%%%%%%%%%%%%
%%     END OF YOUR DOCUMENT     %%
%%%%%%%%%%%%%%%%%%%%%%%%%%%%%%%%%%
\end{document}