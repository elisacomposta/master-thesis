\section*{Abstract in lingua italiana}
I social network online sono spesso studiati per analizzare sia fenomeni individuali che collettivi. 
In questo contesto, i simulatori sono strumenti ampiamente utilizzati per esplorare scenari controllati.
L’integrazione dei Large Language Models (LLM), consente di creare simulazioni più realistiche, grazie alla loro capacità di comprendere e generare linguaggio naturale.

Questo lavoro ha l’obiettivo di studiare il comportamento di agenti LLM in un simulatore di social network.
Gli agenti sono inizializzati con profili realistici e sono calibrati su dati reali relativi alle elezioni politiche italiane del 2022.
Un simulatore social media già esistente è stato esteso introducendo meccanismi per modellare l’opinione degli agenti e per simulare la diffusione di misinformazione.
L’obiettivo è esplorare come gli agenti LLM simulano e conversazioni, interagiscono, ed evolvono le loro opinioni, in diversi scenari.

I risultati mostrano che gli agenti LLM possono generare contenuti coerenti e di formare connessioni con gli altri utenti, costruendo un grafo sociale realistico.
Tuttavia, il tono dei contenuti che generano risulta meno eterogeneo rispetto a quello osservato nei dati reali, in termini di tossicità.
L’evoluzione delle opinioni determinata dagli LLM evolve nel tempo in modo simile a quanto osservato con tradizionali modelli di dinamiche di opinioni.
L'esposizione alla misinformazione non ha un impatto significativo, suggerendo una necessaria modellazione dei modelli cognitivi degli LLM in fase di inizializzazione.
Un'altra limitazione di questo studio riguarda il tempo simulato, che non permette di osservare effetti a lungo termine come l'impatto di diversi algoritmi di raccomandazione.

Nel complesso, gli LLM si dimostrano un potente strumento per simulare il comportamento degli utenti in ambienti sociali, ma ci sono ancora sfide nel rappresentare eterogeneità e pattern comportamentali più complessi.


\vspace{15pt}
\begin{tcolorbox}[arc=0pt, boxrule=0pt, colback=bluePoli!60, width=\textwidth, colupper=white]
    \textbf{Parole chiave:} Simulazione di social media, Large Language Models, Dinamiche dell'opinione, Scienze sociali computazionali
\end{tcolorbox}