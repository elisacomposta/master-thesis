\section*{Abstract in lingua italiana}
I social network online sono spesso studiati per analizzare sia fenomeni individuali che collettivi. 
In questo contesto, i simulatori sono strumenti ampiamente utilizzati per esplorare scenari controllati.
L’integrazione dei Large Language Models (LLM), consente di creare simulazioni più realistiche, grazie alla loro capacità di comprendere e generare linguaggio naturale.

Questo lavoro ha l’obiettivo di studiare il comportamento di agenti LLM in un simulatore di social network.
Gli agenti sono inizializzati con profili realistici e sono calibrati su dati reali relativi alle elezioni politiche italiane del 2022.
Un simulatore social media già esistente è stato esteso introducendo meccanismi per modellare l’opinione degli agenti e per simulare la diffusione di misinformazione.
L’obiettivo è esplorare come gli agenti LLM si comportano nel simulare conversazioni online e quanto sono realistiche le loro interazioni e cambiamenti di opinione.

I risultati mostrano che gli agenti LLM possono generare contenuti coerenti e di formare connessioni con gli altri utenti, costruendo un grafo sociale realistico.
Tuttavia, il loro comportamento risulta meno eterogeneo rispetto a quello osservato nei dati reali, specialmente in termini di tossicità dei contenuti.
L’evoluzione delle opinioni determinata dagli LLM evolve nel tempo in modo simile a quanto osservato con modelli tradizionali di modelli di dinamiche di opinioni.
Per quanto riguarda la misinformazione, non c’è un impatto significativo: gli agenti non sembrano modificare le loro opinioni nemmeno quando esposti ad alti livelli di contenuti falsi.

Questi risultati indicano che gli LLM rappresentano un potente strumento per simulare il comportamento degli utenti in ambienti sociali, ma presentano ancora limiti nel replicare pattern più eterogenei.

\vspace{15pt}
\begin{tcolorbox}[arc=0pt, boxrule=0pt, colback=bluePoli!60, width=\textwidth, colupper=white]
    \textbf{Parole chiave:} qui, le parole chiave, della tesi, in italiano 
\end{tcolorbox}