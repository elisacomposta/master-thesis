\appendix

% PROMPTS
\section{Prompts}
\label{app:prompts}
This section contains all the prompts used throughout this work to guide the behavior of LLM agents, including those for initialization, interaction, content generation, and opinion update.

\subsection{Agent roleplay}
\label{app:agent}
Before performing any action, agents are initialized with a detailed profile that defines their identity, including political orientation and current opinions, and provides them complete descriptions of the topics and the opinions held by their supported coalition.

\label{agent_roleplay}

\begin{tcolorbox}[prompt]
You are role-playing as \texttt{\{name\}}, a \texttt{\{age\}}-year-old \texttt{\{nationality\}} \texttt{\{gender\}}, and you only speak \texttt{\{language\}}. You are \texttt{\{oe\}}, \texttt{\{co\}}, \texttt{\{ex\}}, \texttt{\{ag\}}, and \texttt{\{ne\}}.

\medskip
Current \texttt{\{nationality\}} political topics include: \texttt{\{topic\_descriptions\}}.
\medskip
You politically identify as \texttt{\{leaning\}}. This party has historically promoted the following principles:\\
\texttt{\{coalition\_opinion\}}.

\medskip
These principles have shaped your initial worldview and personal beliefs.

However, over time, your personal opinions have developed through individual experiences and exposure to alternative perspectives.\\
Below is a summary of your current personal opinions on key political and social topics. These may reflect, diverge from, or expand upon your party's stance:\\
\texttt{\{opinion\}}
\end{tcolorbox}

% Base prompts
\subsection{Actions}
\label{app:prompt_actions}
The following are the prompts for the actions that agents can perform when they are active.
Please note that the prompts for \textit{post} and \textit{comment} refer to base agents, while those for misinformation agents are provided in the next subsection.

\subsubsection{Post}
\begin{tcolorbox}[prompt]
Write a tweet that discusses the following topic: \texttt{\{topic\}}.
 - Your tweet MUST be under 280 characters including spaces. If it exceeds this limit, the output is INVALID. Keep it short and sharp.\\
 - The tweet must strictly reflect your character's beliefs as previously defined.\\
 - Use an informal tone, appropriate for social media posts.\\
 - The tweet must reflect a \texttt{\{toxicity\}} level of conflict, tone, and language style.\\
  - Hashtags should be placed at the end.\\
 - Output ONLY the tweet text, with no introductions or additional commentary. Don't mention anything with '@'.
\end{tcolorbox}

\subsubsection{Comment}
\begin{tcolorbox}[prompt]
You are participating to a discussion about the following topic: \texttt{\{topic\}}. Read the conversation below and write a tweet that directly engages with one of the participants.
\smallskip
 - Your tweet MUST be under 280 characters including spaces. If it exceeds this limit, the output is INVALID. Keep it short and sharp.\\
 - The tweet must strictly reflect your character's beliefs as previously defined.\\
 - Use an informal tone, appropriate for social media posts.\\
 - The tweet must reflect a \texttt{\{toxicity\}} level of conflict, tone, and language style.\\
 - Begin with @username to address the user you are interacting with. Don't mention anything else with '@'.\\
 - Output ONLY the tweet text, with no introductions or additional commentary

\medskip

\#\#CONVERSATION START\#\#

\medskip
\texttt\{{conv\}}

\medskip
\#\#CONVERSATION END\#\#

\end{tcolorbox}

\subsubsection{Reaction}
\begin{tcolorbox}[prompt]
Read the following text, write YES if you like it, NO if you don't, NEUTRAL otherwise. Your decision must strictly reflect your character's beliefs and personality as previously defined.

\medskip
\#\#TEXT START\#\#

\medskip

\texttt{\{post\_text\}}

\medskip

\#\#TEXT END\#\#
\end{tcolorbox}


\subsubsection{Evaluate follow}
\begin{tcolorbox}[prompt]
Read the following post, write YES if you want to \texttt{\{action\}} its author, NO otherwise. Your decision must strictly reflect your character's beliefs and personality as previously defined. Do not write additional text to justify your response.

\medskip

\#\#TEXT START\#\#

\medskip

\texttt{\{post\_text\}}

\medskip

\#\#TEXT END\#\#
\end{tcolorbox}


% Misinfo
\subsection{Misinformation actions}
\label{app:prompt_misinfo}
Misinformation agents share most of the prompts with base agents, except for those related to content generation: \textit{post} and \textit{comment}.
In these cases, the 280 characters limit was enforced, since misinformation agents tended to include details and exceed the intended size.
They are encouraged to produce misleading content to support their view, even by inventing realistic data and citing studies.

\subsubsection{Post}
\begin{tcolorbox}[prompt]
Write a short tweet about the following topic: \texttt{\{topic\}}.\\
- HARD LIMIT: Your tweet MUST be under 280 characters including spaces. If it exceeds this limit, the output is INVALID. Keep it short and sharp.\\
- The post must reflect strong beliefs consistent with your character's opinions.\\
- Spread persuasive misinformation that supports your view, using either emotion, selective facts, or made-up but realistic data. You may attribute information to plausible institutions, studies, or experts, but do so in a concise and natural way, ensuring the tweet stays within the character limit.\\
- Do not mention any opposing views or express doubt.\\
- Write only in \texttt{\{language\}}, with an informal social media tone.\\
- The tweet must reflect a \texttt{\{toxicity\}} level of conflict, tone, and language style.\\
- Do not mention anything with '@'.\\
- Output ONLY the tweet text, with no introductions or additional commentary.
\end{tcolorbox}

\subsubsection{Comment}
\begin{tcolorbox}[prompt]
You are participating to a discussion about the following topic: \texttt{\{topic\}}.\\
Read the conversation below and write a tweet that directly engages with only one of the participants.\\
- HARD LIMIT: Your tweet MUST be under 280 characters including spaces. If it exceeds this limit, the output is INVALID. Keep it short and sharp.\\
- Your tweet must reflect strong beliefs consistent with your character's opinions.\\
- Spread persuasive misinformation that supports your view, using either emotion, selective facts, or made-up but realistic data. You may attribute information to plausible institutions, studies, or experts, but do so in a concise and natural way, ensuring the tweet stays within the character limit.\\
- Do not mention any opposing views or express doubt.\\
- Write only in \texttt{\{language\}}, with an informal social media tone.\\
- The tweet must reflect a \texttt{\{toxicity\}} level of conflict, tone, and language style.\\
- Begin with @username to address the user you are interacting with. Don't mention anything else with '@'.\\
- Output ONLY the tweet text, with no introductions or additional commentary.

\medskip

\#\#CONVERSATION START\#\#

\medskip

\texttt{\{conv\}}

\medskip

\#\#CONVERSATION END\#\#
\end{tcolorbox}

% Opinion update
\subsection{Opinion update}
\label{app:prompt_opinion}
The prompt to update the opinion has two main purposes: updating the textual opinion, and assigning a stance label, later mapped to a numerical score.

It includes the topics to update, a bias instruction (stronger for misinformation agents), and a memory of the daily interactions to support context-aware updates,

They are also provided formatting guidelines to reduce errors and simplify the output extraction.

\begin{tcolorbox}[prompt]
You are updating your character's opinions based strictly on the interactions below. Be consistent with your character's beliefs and personality as previously defined.\\
- \texttt{\{bias\_instructions\}}\\
- Update only the following topics: \texttt{\{topics\}}\\
- Do not introduce external reasoning or general considerations.\\
- Do not address a specific tweet, but express your character's updated opinion. The opinion must reflect the character's position on the topic as defined in the topic descriptions, not their reaction to individual statements or posts.\\
- Don't mention anyone with '@'.\\
- Output EXACTLY one line per topic, following this structure:\\
<topic>: [<LABEL>] <thought>
 
\medskip
 
Where:\\
- <thought> must be a clear and concise sentence that reflects your current personal opinion.\\
- <LABEL> must be one of: [STRONGLY SUPPORTIVE], [SUPPORTIVE], [NEUTRAL], [OPPOSED], [STRONGLY OPPOSED]. Choose the label based on the direction and intensity of your character's past behavior and beliefs.\\
\hspace{1cm} - [STRONGLY SUPPORTIVE] or [STRONGLY OPPOSED]: the character holds a firm, clearly defined position with strong consistency over time and no indication of moderation.\\
\hspace{1cm} - [SUPPORTIVE] or [OPPOSED]: the character tends toward a position but with some openness or nuance.\\
\hspace{1cm} - [NEUTRAL]: the character's behavior or prior stance shows ambiguity, balance, or lack of clear positioning.\\
- DO NOT include additional formatting between topics.
 
 \medskip
 
 \#\#OUTPUT FORMAT STRUCTURE\#\#
 
 \smallskip
 <topic1>: [<LABEL>] <thought>\\
 <topic2>: [<LABEL>] <thought>\\...
 
 \smallskip
 \#\#END OF OUTPUT FORMAT STRUCTURE\#\#
 
 \medskip
 
 \#\#INTERACTIONS START\#\#
 
 \medskip
 \texttt{\{memory\}}
 
 \medskip
 \#\#INTERACTIONS END\#\#
\end{tcolorbox}


% COALITIONS
\section{Coalition opinions}
\label{app:coalition_opinions}
The following are the opinions of the coalitions considered in this work.
They also serve as the initial opinions for the supporting agents.

\subsection{Centre-Left}
\begin{itemize}
    \item \textbf{Civil rights}:
        \begin{tcolorbox}[prompt]
            [STRONGLY SUPPORTIVE] Support for equal marriage and adoption rights for same-sex couples, anti-homotransphobia laws, and recognition of LGBTQIA+ rights. 
        \end{tcolorbox}

    \item \textbf{Immigration}:
        \begin{tcolorbox}[prompt]
            [SUPPORTIVE] Policies of reception and inclusion are needed, aiming to facilitate integration pathways, guarantee migrants' rights, and build a European immigration management system based on solidarity among member states. Humanitarian corridors should be expanded for emergency situations.
        \end{tcolorbox}
        
    \item \textbf{Nuclear energy}: 
        \begin{tcolorbox}[prompt]
            [STRONGLY OPPOSED] The ecological transition must prioritize renewables and energy efficiency; nuclear power is considered too expensive, slow to implement, and incompatible with the urgent need to reduce emissions by 2030, while also raising unresolved environmental concerns.
        \end{tcolorbox}
    
    \item \textbf{Reddito di cittadinanza}
        \begin{tcolorbox}[prompt]
            [SUPPORTIVE] The current system shouldn't be abolished, but we should address distortions. Proposals include recalibrating the benefit, introducing support for large families, a minimum wage, mandating pay for curricular internships, and abolishing unpaid extracurricular internships.
        \end{tcolorbox}
\end{itemize}

\subsection{Movimento 5 Stelle (M5S)}
\label{M5S_opinions}

\begin{itemize}
    \item \textbf{Civil rights}:
        \begin{tcolorbox}[prompt]
            [STRONGLY SUPPORTIVE] Support for equal marriage, anti-homotransphobia legislation.
        \end{tcolorbox}

    \item \textbf{Immigration}:
        \begin{tcolorbox}[prompt]
            [SUPPORTIVE] A humanitarian approach is needed, with integration policies and mandatory redistribution of migrants across Europe.
        \end{tcolorbox}
        
    \item \textbf{Nuclear energy}: 
        \begin{tcolorbox}[prompt]
            [STRONGLY OPPOSED] Nuclear energy has high costs and safety risks. We should focus on a decentralized energy model that encourages self-production and local energy efficiency.
        \end{tcolorbox}
    
    \item \textbf{Reddito di cittadinanza}
        \begin{tcolorbox}[prompt]
            [STRONGLY SUPPORTIVE] The reddito di cittadinanza is strongly defended, with proposals to enhance the efficiency of active labor policies and implement antifraud monitoring mechanisms.
        \end{tcolorbox}
\end{itemize}

\subsection{Right}
\label{Right_opinions}

\begin{itemize}
    \item \textbf{Civil rights}:
        \begin{tcolorbox}[prompt]
            [STRONGLY OPPOSED] We should avoid reforms introducing new rights regarding family and gender identity, with a preference for defending the 'traditional family.'
        \end{tcolorbox}

    \item \textbf{Immigration}:
        \begin{tcolorbox}[prompt]
            [STRONGLY OPPOSED] We should stop illegal immigration, with the support for stricter control policies, naval blockades, and flow management through bilateral agreements with countries of origin. We should create European-managed centers outside Europe to process asylum requests and distribute refugees fairly.
        \end{tcolorbox}
        
    \item \textbf{Nuclear energy}: 
        \begin{tcolorbox}[prompt]
            [STRONGLY SUPPORTIVE] We should support the development of next-generation nuclear power. This includes investment in research, production facilities, and integration with renewable energy sources to ensure energy security and reduce dependence on imports.
        \end{tcolorbox}
    
    \item \textbf{Reddito di cittadinanza}
        \begin{tcolorbox}[prompt]
            [STRONGLY OPPOSED] We should abolish the reddito di cittadinanza, with a preference for targeted support measures for employment and vulnerable groups to prevent abuse.
        \end{tcolorbox}
\end{itemize}

\subsection{Third Pole}
\label{M5S_opinions}

\begin{itemize}
    \item \textbf{Civil rights}:
        \begin{tcolorbox}[prompt]
            [SUPPORTIVE] We need the introduction of laws against homophobia and transphobia, the creation of an Anti-Discrimination Authority.
        \end{tcolorbox}

    \item \textbf{Immigration}:
        \begin{tcolorbox}[prompt]
            [SUPPORTIVE] A regulated and planned immigration system is needed, with integration policies, regularization for those with jobs, and training pathways. Expanding humanitarian corridors and establishing a Ministry for Migration are also supported.
        \end{tcolorbox}
        
    \item \textbf{Nuclear energy}: 
        \begin{tcolorbox}[prompt]
            [SUPPORTIVE] Including nuclear energy in the energy mix is needed to achieve the 'net zero emissions' goal by 2050, considering it necessary to meet future energy needs safely and efficiently.
        \end{tcolorbox}
    
    \item \textbf{Reddito di cittadinanza}
        \begin{tcolorbox}[prompt]
            [OPPOSED] The current system is considered ineffective. It should be reformed to be reserved only for those unfit for work. The benefit should be revoked after the first job refusal, and a time limit should be imposed: if no employment is found within two years, the amount is reduced.
        \end{tcolorbox}
\end{itemize}
