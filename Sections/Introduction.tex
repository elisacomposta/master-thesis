\section{Introduction}
\label{sec:introduction}

% General context
In the last years, online social networks have gradually become fundamental part of everyday life of millions of people.
They are no longer just platforms for communication, but real digital spaces where users express their emotions and shape their opinions and behaviors \cite{bakshy2015}.
This makes them an interesting tool not only for studying individual dynamics in online interactions, but also for exploring complex collective phenomena, such as polarization, content diffusion, and other social processes \cite{vosoughi2018spread}.

% Simulations
\medskip
One powerful way to investigate these dynamics is through the use of simulators.
These tools make it possible to recreate controlled virtual environments where it's possible to test specific scenarios, compare different strategies and observe the evolution of the behavior of users, even under conditions that would be difficult, or ethically problematic, to reproduce in real world.
For instance, it's possible to analyze the impact of the diffusion of toxic content or fake news, or evaluate how different recommendation algorithms influence user behavior, without the need to interfere with real platforms.

However, building a realistic simulation of a social network is a complex challenge.
The behaviors that emerge from these systems are shaped by of many individual human factors, which are often hard to predict or formalize.
The human nature of interactions introduces variability, ambiguity, and contextual information, making the development of such systems particularly difficult.


% ABMs
\medskip
To study the complex phenomena on social networks, a widely used approach the Agent-Based Modeling (ABM).
This system allows the simulation of complex systems by describing the behavior of individual agents, each following a set of predefined rules.
ABMs are useful for modeling online social media because they include many heterogeneous agents that interact with each other \cite{gausen2021can}.
Through the interaction of agents, it's possible to observe unexpected collective dynamics and study the relationship between phenomena at the micro level (individual) and macro level (system).

In traditional models, agents are described by relatively simple rules, guiding their behavior in a deterministic or probabilistic way.
Even though this approach led to some interesting results, it shows limitations when it comes to replicating the complexity of human behavior, which includes language, emotions and context \cite{conte2014agent, törnberg2023evaluate}.

In this scenario, Large Language Models (LLMs), linguistic models able to generate and understand text in natural language, represent a promising evolution.
Integrating LLMs as agents in ABM simulators enables more realistic behaviors, as they can coherently replicate conversations, opinions, emotions and different interaction strategies \cite{park2023genagents}.

Studies in this direction have shown promising results, suggesting that LLMs have a great potential for simulating human behavior and deserve to be further explored \cite{gao2023s3socialnetworksimulationlarge, törnberg2023evaluate, rossetti2024ysocialllmpoweredsocial}.


% Obiettivo del lavoro
\medskip
This work is based on \textit{Y} \cite{rossetti2024ysocialllmpoweredsocial}, a social media simulator that replicates online social platforms in a controlled environment.
The framework can simulate different scenarios and uses LLM agent, designed to behave like real users: they read, post and interact with other users.

The main contribution of this work is to extend the original \textit{Y} system in three main directions:
\begin{itemize}
    \item Initializing agents with real-world data \cite{pierri2023ita}, collected around the 2022 Italian elections, to improve the realism of their behavior.
    \item Introducing explicit management of agents' opinions on specific political topics, allowing them to evolve over time
    \item Defining a new category of agents who share misleading content to support their views.
\end{itemize}

Information disorders such as disinformation and misinformation are increasingly relevant phenomena, with direct effects on users' opinions and the online dynamics.
Simulating their spread in a virtual environment can help analyze their impact and explore possible prevention and mitigation strategies.

The main goal of this work is to explore the behavior of LLM agents in the simulator, evaluating their potential to model social dynamics, and identifying the limitations of this approach.
The analysis focuses on opinion evolution, the impact of misinformation, user interactions, and the emerging network structure.


% Più info su contesto politico delle simulazioni ?
% Più info su disinformation?



% Methods (workflow, network init e come si forma, interazioni, opinioni aggiornate da LLM e anche math per confronto -> focus sul lavoro
% Intro conclusion delle contributions del lavoro (estende framework, e cosa analizzo)
% Structure of the thesis