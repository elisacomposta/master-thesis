\section{Introduction}
\label{sec:introduction}

% General context
In the last years, online social networks have gradually become fundamental part of everyday life of millions of people.
They are no longer just platforms for communication, but real digital spaces where users express their emotions and shape their opinions and behaviors.
This makes them an interesting mean not only for studying individual dynamics in online interactions, but also for exploring complex collective phenomena, such as polarization, content diffusion, and other social processes.

% Simulations
\medskip
One powerful way to investigate these dynamics is through the use of simulators.
These tools make it possible to recreate controlled virtual environments where it's possible to test specific scenarios, compare different strategies and observe the evolution of the behavior of users, even under conditions that would be difficult, or ethically problematic, to reproduce in real world.
For instance, it's possible to analyze the impact of the diffusion of toxic content or fake news, or evaluate how different recommendation algorithms influence user behavior, without the need to interfere with real platforms.

However, building a realistic simulation of a social network is a complex challenge.
The behaviors that emerge from these systems are shaped by of many individual human factors, which are often hard to predict or formalize.
The human nature of interactions introduces variability, ambiguity, and contextual information, making the development of such systems particularly difficult.


% ABMs
\medskip
To study the complex phenomena on social networks, a widely used approach the Agent-Based Modeling (ABM).
This system allows the simulation of complex systems by describing the behavior of individual agents, each following a set of predefined rules.
Through the interaction of agents, it's possible to observe emerging collective dynamics and study the relationship between phenomena at the micro level (individual) and macro level (system).

In traditional models, agents are described by relatively simple rules, guiding their behavior in a deterministic or probabilistic way.
Even though this approach led to some interesting results, it shows limitations when it comes to replicating the complexity of human behavior, which includes language, emotions and context.

In this scenario, Large Language Models (LLMs), linguistic models able to generate and understand text in natural language, represent a promising evolution.
Integrating LLMs as agents in ABM simulators enables more realistic behaviors, as they can coherently replicate conversations, opinions, emotions and different interaction strategies.

Studies in this direction have shown promising results, suggesting that LLMs have a great potential for simulating human behavior and deserve to be further explored.


% Obiettivo del lavoro (Basato su Y digital twin, propongo estensione per opinione e misinfo, analizzo comportamento LLMs, evoluzione opinioni e impatto misinfo
% Methods (workflow, network init e come si forma, interazioni, opinioni aggiornate da LLM e anche math per confronto -> focus sul lavoro
% Intro conclusion delle contributions del lavoro (estende framework, e cosa analizzo)
% Structure of the thesis